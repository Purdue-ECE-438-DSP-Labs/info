\documentclass[letterpaper, 11pt]{article}
% \thispagestyle{empty}
\usepackage{amsmath}
\usepackage{fullpage}
\usepackage{hyperref}
%\usepackage{graphicx}
% \usepackage{epsfig}
% \usepackage{epstopdf}
%  \usepackage{graphicx}
%  \usepackage{float}
%\DeclareGraphicsRule{.tif}{png}{.png}{`convert #1 `dirname #1`/`basename #1 .tif`.png}
%\renewcommand{\familydefault}{\rmdefault}
\usepackage{times}
\usepackage{tabto}
\newcommand\mytab{\tab \hspace{-2cm}} 
 
\begin{document}
\begin{center}
\huge \textbf{ ECE 438 Digital Signal Processing} \\
 \textbf{Lab Syllabus Spring 2022 }\\
\medskip
\normalsize Labs can be found on \href{https://github.com/Purdue-ECE438-Labs/}{the Github page}.
\end{center}

\section*{TA information}
\begin{enumerate}
\item[] Yang Cheng \mytab email: cheng159@purdue.edu\\
Tianyu Li \mytab email: li2516@purdue.edu\\
William Tran \mytab email: tran168@purdue.edu
\end{enumerate}



\textit{\textbf{Office hours}} - Office hours will be posted on the course website.

\section*{Lab Content}
The experiments in this course generally build upon material covered in lecture. Each experiment describes one or more theoretical concepts, and then presents exercises to implement the concepts within the Python environment. All of the lab content is directly relevant to the exercises, so it is important that each section be read and understood before performing the corresponding exercises. As a rule, the entire lab should be read before your section meeting, it will prepare you for the pre-lab quiz each week and help you complete the lab faster. If the lab work is not completed during the lab period, the remaining work should be done during open lab time or at home.

\section*{Prerequisite}
Programming experience. Students who have little to no experience in programming find it difficult to do 438 labs.

\section*{Attendance}
Students must attend every laboratory as health permits. Masks and face shields will be required for attendance at the labs. If you are feeling ill or suspect that you have come in contact with someone who has tested positive for COVID-19, do not attend the lab and notify your TA (not the instructor) and they will give you instructions on completing the quiz and lab. If you need to miss a lab due to an academic/business trip or job interview you should notify your TA before the missing lab. Students are not allowed to leave before the end of their lab section unless they have completed the lab or they have permission from their TA.


\newpage
\section*{Laboratory Components}
\begin{enumerate}
\item[1.] Weekly Pre-Lab Quiz:
\begin{itemize}
\item[$-$] 30\% of total lab grade (Your lowest quiz grade will be dropped.)
\item[$-$] Each quiz is worth 10 points and is based on the lab content that will be covered on that day.
\item[$-$] You will be given 15 minutes to complete the quiz.
\item[$-$] If a student is more than 10 minutes late for the quiz, they will need to take a makeup quiz at a later date. The makeup quiz time must be coordinated with your TA. 
\item[$-$] A student can take up to 2 makeup quizzes. Makeup quizzes due to illness or COVID-19 related absences will not count against the 2 makeup quiz limit.
\end{itemize}
\item[2.] Laboratory Report:
\begin{itemize}
\item[$-$] 70\% of total lab grade
\item[$-$] Lab reports will consist of a cover sheet/grading rubric, plots, written work, and code specified in the lab document.
\item[$-$] Lab reports are graded out of 100 points
\item[$-$] Late reports will be penalized at 10 points per day.
\item[$-$] Two or three students make one group and each group submits one report in PDF. They do so by opening their Jupyter notebook and saving it as PDF by Ctrl + P (Windows) or Cmd + P (MacOS), then uploading it to Gradescope. And they need to make sure that all cell blocks, plots are correctly displayed in the rendered PDF.
\item[$-$] If a student needs to miss a lab for a valid reason, then she/he still needs to do the lab and submit the report by the end of the day (11:59 pm) of their next lab session. 
If the student who is missing the lab and her/his teammate can collaborate during the week, then they are allowed to submit one lab report. Otherwise, they each submit individual reports.
\item[$-$] No formal reports are required, you need only to complete the exercises and answer the questions within the provided Jupyter Notebook (.ipynb) file. These exercises are in \textbf{bold} font in the file.
\item[$-$] Discussions about concepts and equations are encouraged, but sharing/copying code is prohibited.
\item[$-$] At the end of the semester, if the average proportion, as indicated on the lab cover sheets, of your participation for lab reports is below 40\% (or 25\% for groups of 3), your lab grade will be dropped by one level. (e.g., if you get B for labs, you will be given a C.)
\end{itemize}
\item[3.] Code Submission
\begin{itemize}
\item[$-$] Each student is required to submit her/his own code on Github by the end of the day (11:59 pm) of their next lab session. 
\item[$-$] Upload all the code back onto the Github repository using git push \newline
* Tue 8:30 – 11:20 (Sec 3), Tue 11:30 – 2:20 (Sec 5), Thu 11:30 – 2:20 (Sec 2). 
\item[$-$] Do not look at one another's code. Looking at another's code and making yours similar is also considered cheating. Any kind of cheating will result in failure of the 438 lab.
\end{itemize}

\end{enumerate}

\newpage
\section*{Cheating}
Don’t do it. Specifically, if you have access to a previously graded lab report, do not even look at it. Certainly do not use someone else’s code to perform the exercises. Some examples of cheating would be:
\begin{itemize}
\item[$-$] Downloading large portions of code from the internet.
\item[$-$] Looking at other student’s completed code.
\item[$-$] Writing code while looking at other’s code.
\item[$-$] Copying written work from others.
\item[$-$] Copying your lab partner's code.
\item[$-$] Submitting report or code that includes someone else's work.
\item[$-$] Requesting a re-grade of lab report that has been altered.
\item[$-$] Looking at another student’s quiz.
\end{itemize}

\noindent
\textbf{\underline{Any kind of cheating will result in a failing lab grade}}. This would require the student to retake the lab in a subsequent semester. All occurrences of academic dishonesty will be reported to the Assistant Dean of Students and copied to the ECE Assistant Head for Education. If there is any question as to whether a given action might be construed as cheating, please see the professor or the TA before you engage in any such action.

\section*{Classroom Guidance Regarding Protect Purdue}

The \href{https://protect.purdue.edu/plan/}{Protect Purdue Plan}, which includes the \href{https://protect.purdue.edu/pledge/?_ga=2.210401429.1213937682.1590527202-1814553957.1589408073}{Protect Purdue Pledge}, is campus policy and as such all members of the Purdue community must comply with the required health and safety guidelines. Required behaviors in this class include: staying home and contacting the Protect Purdue Health Center (496-INFO) if you feel ill or know you have been exposed to the virus, properly wearing a mask \href{https://protect.purdue.edu/updates/face-covering-protocols/}{in classrooms and campus building}, at all times (e.g., mask covers nose and mouth, no eating/drinking in the classroom), disinfecting desk/workspace prior to and after use, maintaining appropriate social distancing with peers and instructors (including when entering/exiting classrooms), refraining from moving furniture, avoiding shared use of personal items, maintaining robust hygiene (e.g., handwashing, disposal of tissues) prior to, during and after class, and following all safety directions from the instructor.

Students who are not engaging in these behaviors (e.g., wearing a mask) will be offered the opportunity to comply. If non-compliance continues, possible results include instructors asking the student to leave class and instructors dismissing the whole class. Students who do not comply with the required health behaviors are violating the University Code of Conduct and will be reported to the Dean of Students Office with sanctions ranging from educational requirements to dismissal from the university.

Any student who has substantial reason to believe that another person in a campus room (e.g., classroom) is threatening the safety of others by not complying (e.g., not wearing a mask) may leave the room without consequence. The student is encouraged to report the behavior to and discuss next steps with their instructor. Students also have the option of reporting the behavior to the \href{https://www.purdue.edu/odos/osrr/}{Office of the Student Rights and Responsibilities}. See also \href{https://catalog.purdue.edu/content.php?catoid=7&navoid=2852#purdue-university-bill-of-student-rights}{Purdue University Bill of Student Rights}. 

\end{document}


